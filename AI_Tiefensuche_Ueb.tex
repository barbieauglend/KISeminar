\documentclass[fleqn]{article}

\usepackage[T1]{fontenc}
\usepackage[utf8]{inputenc}
\usepackage[ngerman]{babel}
\usepackage{amsmath,amsfonts,amsthm,graphicx}
\usepackage{sectsty}
\usepackage{booktabs}
\usepackage{algpseudocode}
\usepackage{algorithm}
\usepackage{listings}
\usepackage{tikz, pgf, verbatim}
\usepackage{float}

\numberwithin{equation}{section}
\numberwithin{figure}{section}
\numberwithin{table}{section}

\pgfdeclarelayer{background}
\pgfsetlayers{background,main}

\usetikzlibrary{calc, arrows, shapes}

\renewcommand{\thesubsubsection}{(\alph{subsubsection})}

\title{Überblick Künstliche Intelligenz}
\author{Thaís Moreira Hamasaki}
\date{\today}

\begin{document}

  \lstset{language=Haskell,frame=single,basicstyle=\small\ttfamily,numbers=left,firstnumber=1}
  \lstset{language=Prolog,frame=single,basicstyle=\small\ttfamily,numbers=left,firstnumber=1}

\section{Aufgabe 01 - Tiefen- und Breitensuche}

Gegeben sei der folgende Graph:\\
\bigskip

\tikzstyle{vertex}=[circle,fill=black!25,minimum size=20pt,inner sep=0pt]
\tikzstyle{selected vertex} = [vertex, fill=red!24]
\tikzstyle{edge} = [draw,thick,--]

\begin{tikzpicture}[scale=1.8, auto,swap]
    \foreach \pos/\name in {{(1,2)/c}, {(6,2)/j},
                            {(0,0)/b}, {(2,0)/d}, {(4,0)/f}, {(5,0)/h}, {(7,0)/k},
                            {(0,-2)/a}, {(2,-2)/e}, {(4,-2)/g}, {(5,-2)/i}, {(7,-2)/l}}
                            \node[vertex] (\name) at \pos {$\name$};
    \foreach \source/ \dest in {c/j, b/c, d/c ,f/c, f/j, h/j, k/j,
                                         b/d, d/c, d/f, f/h, h/k,
                                         a/b, e/d, g/f, g/d, i/h, l/k, i/k, l/h,
                                         a/e, e/g, e/f, i/l}
                            \path[edge] (\source) -- node[] {} (\dest);
\end{tikzpicture}
\bigskip

Geben Sie für den obigen Graphen die Nummern $t(v)$ und $b(v)$. Starten  Sie  die  Suche  bei  Knoten $a$.  Die Adjazenzlisten der Knoten seien alphabetisch sortiert.\\


\textbf{Lösung:}\\
\bigskip
\begin{tabular}{c|c c c c c c c c c c c c c}
  v & a & b & c & d & e & f & g & h & i & j & k & l \\
  \hline
  $t(v)$ & 1 & 2 & 3 & 4 & 5 & 6 & 7 & 8 & 9 & 11 & 10 & 12\\
  $t(v)$ & 1 & 2 & 4 & 5 & 3 & 6 & 7 & 9 & 11 & 8 & 10 & 12\\
\end{tabular}

\section{Aufgabe 02 - Topologisches Sortieren}

Ermitteln Sie für den folgenden gerichteten Graphen G = (V, A) eine topologische Sortierung der Knoten.\\
\bigskip

\tikzstyle{vertex}=[circle,fill=black!25,minimum size=20pt,inner sep=0pt]
\tikzstyle{selected vertex} = [vertex, fill=red!24]
\tikzstyle{edge} = [draw,thick,->]

\begin{tikzpicture}[scale=1.8, auto,swap]
    \foreach \pos/\name in {{(0,3)/a}, {(1,3)/c}, {(3,3)/f}, {(4,3)/h},
                            {(2,2)/e}, {(5,2)/j},
                            {(0,1)/b}, {(1,1)/d}, {(3,1)/g}, {(4,1)/i},
                            {(5,0)/k}, {(6,0)/l}}
                            \node[vertex] (\name) at \pos {$\name$};
    \foreach \source/ \dest in {a/c, f/c, f/h, e/c, e/f, e/d, g/f, h/j, g/h,
                                b/d, g/i, i/j, i/k, l/k, c/d, e/g, h/i}
                            \path[edge] (\source) -- node[] {} (\dest);
\end{tikzpicture}

\textbf{Lösung:}\\
\bigskip

$(a, b, e, g, f, c, d, i, h, j, l, k)$ ist damit eine von vielen möglichen topologischen Ordnungen.

\section{Aufgabe 03 - Algorithm}

Vervollständigen Sie den Algorithmus für Hill Climbing:

\begin{algorithm}
\caption{Hill Climbing Algorithm}
\begin{algorithmic}[1]
\Function{Hill-Climbing}{problem} \State \textbf{returns} a solution state
\INPUT
\Statex $problem$ \Comment Ein Problem
\Statex $current$  \Comment Ein Knoten
\Statex $next$  \Comment Ein Knoten
\State $current \gets Make-Node(Initial-State[problem])$
\Loop
\State $next \gets$ $A$ Wert aus $current$
\If{$VALUE[B] C VALUE[D]$} \Return current
\EndIf
\State $current \gets next$
\EndLoop
\EndFunction
\end{algorithmic}
\end{algorithm}

\textbf{Lösung:}\\
\bigskip
A: größter, B: next, C: <, D: current


\end{document}
